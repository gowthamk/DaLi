\section{Distributed Instantiation}
\label{sec:system-model}



The operational semantics of \name describe a concurrency model based
on versions and branches for computations on replicated data, without
referring to any particular instantiation. A useful instantiation of
the model is for a distributed system with multiple nodes, where each
each node hosts a \emph{replica} of the data. Thus, each node
instantiates a branch. The node responds to user requests to perform
updates to the replica, and publishes the results on its local branch
as newer versions. The node also stores (a prefix of) the branching
history - a knowledge about the system that it accumulates over time.
In response to the publication of a new local version, it contacts
other nodes to obtain their latest local versions, in the process
updating its knowledge about the branching history. If a later version
($v_j$) on a remote branch ($b_j$) is mergeable with the latest
version ($v_i$) on the local branch ($b_i$), i.e., if $v_j \not\preceq
v_i$ and LCA of $b_i$ and $b_j$ is unique, the node initiates the
merge process with the remote node, where in it lets the remote node
know that a new version ($v_i'$) later than $v_i$ is being created on
$b_j$ by merging version $v_j$ of $b_j$. This prevents the remote node
from merging versions $v_i$ or earlier of $b_i$, and creating a
criss-cross branching structure. The later merge of $b_i$ into $b_j$
by the remote node will only consider versions $v_i'$ or later. The
merge operation is a function over participating branches, and occurs
in a single time step, extending one of the branches. Concurrent
merges that do not seek to extend a branch participating in the
current merge operation are allowed. The resultant history of merges
will still be linearizable. The example in
Fig.~\ref{fig:instantiation-1} illustrates. Here, in the time step
$t=1$, nodes $n_1$ and $n_2$ communicate to merge branch $b_2$ in
$b_1$. Concurrently, $n_1$ and $n_3$ can also communicate to merge
$b_2$ into $b_3$. Likewise, at $t=1$, merges from $b_1$ into $b_2$ and
$b_1$ into $b_3$ happen concurrently. The resultant history is
equivalent to performing the merges in the order as they were
described above.


