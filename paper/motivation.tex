\section{Motivation}

In this section, we motivate mergeable data types and the \name
programming model via a series of examples.

\subsection{Counter}

\begin{figure}

\begin{subfigure}[b]{0.4\textwidth}
  \begin{ocaml}
    module Counter: sig
      type t
      val add: int -> t -> t
      val mult: int -> t -> t
      val read: t -> int
    end = struct
      type t = int
      let add x v = v + (abs x)
      let mult x v = v * (abs x)
      let read v = v
    end
  \end{ocaml}

\caption{\C{Counter} library in OCaml}
\label{fig:counter-adt}
\end{subfigure}
\begin{subfigure}[b]{0.56\textwidth}
  \begin{ocaml}
    module Counter: sig
      type t
      type eff 
      val add: int -> t-> eff
      val mult: int -> t -> eff
      val apply: eff -> t -> t
      val read: t -> int
    end = struct
      type t = int
      type eff = Add of int
      let add x v = Add (abs x)
      let mult x v = Add (v * (abs x - 1))
      let apply (Add x) v = x + v
      let read v = v
    end
  \end{ocaml}

\caption{\C{Counter} library re-engineered for effect-based replication}
\label{fig:counter-rdt}
\end{subfigure}
% \begin{minipage}{0.62\textwidth}
%   \begin{ocaml}
%     module Counter = struct
%       include Counter
%       type t = counter [@@deriving persistence]
%       let merge old v1 v2 = old + (v1 - old) 
%                                 + (v2 - old)
%     end
%   \end{ocaml}
% \end{minipage}

% \caption{\C{Counter} library in OCaml equipped with a \C{merge}
% function. The \C{deriving} syntax~\cite{ppx-deriving} is processed by
% \name's meta-programming library to automatically derive a persistent
% variant of the \C{Counter}}

% \label{fig:counter}
\end{figure}

We will start with a simple example of a counter that nonetheless
demonstrates the benefits of our approach. An OCaml implementation of
the \C{Counter} library is shown in Fig.~\ref{fig:counter-adt}.
\C{Counter} supports two (update) operations - \C{add} and \C{mult} -
that let a non-negative integer value to be added to the counter, and
multiply the counter, respectively. Observe that the library is
written in an idiomatic style in OCaml; any OCaml programmer faced
with the task of implementing the counting functionality would
presumably write the same.  

The counter implementation of Fig.~\ref{fig:counter-adt} serves the
application well as long as it is used in-memory on a single machine.
However, as the application grows in complexity and scales beyond a
single machine, its state may also need to be replicated and
persisted. Unfortunately, replication doesn't come for free, and
existing techniques require significant re-engineering of the
application and its libraries to support replication. A method often
adopted~\cite{crdts, pldi15, gotsman-popl16} is to re-define the operations to
return \emph{effects} instead of values.  An \emph{effect} is a tag that
identifies the operation to be applied uniformly at all
replicas to incorporate the effects of the original operaton. For
instance, \C{Counter.add x v} operation returns \C{Add x} effect,
which, when applied at a replica (see \C{apply}), adds \C{x} to the 
local counter value.  Fig.~\ref{fig:counter-rdt} shows the
\C{Counter} library with operations re-defined to returns effects.
Such re-engineering has to be done carefully, lest an operation may
may end up causing an unintended effect at a remote replica. For
instance, one may incorrectly define \C{mult x v} operation as
generating a \C{Mult x} effect, for it is easy to overlook the fact
that \C{apply (Mult x)} affects different replicas differently
depending on the counter value. Thus, reasoning about the correctness
of effects often involves appeals to system-specific artifacts, such
as replicas, which don't normally manifest in the program.  While
checking the effects (or rather their application via \C{apply}) for
commutativity may help catch some of these oversights, commutativity
is a non-local property, and automatically verifying it is difficult.
Besides introducing the possibility of new bugs, re-engineering
libraries to return effects has obvious software engineering problems
in the sense that it breaks backwards compatibility with the existing
code.

Another downside with effect-generating replicated data type libraries
is that they don't compose. For instance, let us say that an
application wants to use two replicated counters, $c_1$ and $c_2$,
bound by an invariant $c_2 \ge c_1$, which the application duly
enforces when updating $c_2$ and $c_1$. Nonetheless, it may still
witness anamolous states that violate the invariant because updates to
$c_1$ and $c_2$ are applied independently in any order. For example,
if a replica increments both the counters by 1 each, the \C{read} at
another replica, where $(c_1,c_2) = (2,2)$ may witness the increment
to $c_1$, but not $c_2$, thus violating the invariant. The intention
to atomically apply both the effects cannot be captured in the absence
of external mechanisms to compose effects (such as effect-based
transactions~\cite{pldi15}).  

Persistence adds another dimension to the problem. Functional data
structures are often large in-memory linked structures that admit
functional updates by creating newer versions that share most of their
structure with previous versions. Sharing is the key to efficiency,
without which every functional update takes time that is at least
linear in the size of the data structure. Unfortunately, the
straightforward way to persist a data structure on disk in a
machine-agnostic format (e.g., JSON) requires serialization, which is
linear in the size of the data structure. An alternative is to
maintain a separate on-disk copy of the data and administer it via a
database system. The downside is that the programmer is now required
to reason in terms of a lower-level abstraction (e.g., a relational
data model) in order to maintain coherence between in-memory and
on-disk representations. Safety guarantees over the in-memory data do
not carry over to the disk, which introduces additional complications
and the possibility of new bugs. Thus, adding persistence to
applications while preserving high-level safety gurantees offered by
the language abstraction is a non-trivial problem.

\name lets programmers add replication and persistence to their
applications without having to re-engineer their abstractions or
reason about lower-level artifacts. Here is how you write a mergeable
counter in \name:
\begin{figure}[!h]
  \begin{ocaml}
    module Counter = struct
      include Counter
      type mergeable_t = t [@@deriving persistence]
      let merge old v1 v2 = old + (v1 - old) + (v2 - old)
    end
  \end{ocaml}
\end{figure}


