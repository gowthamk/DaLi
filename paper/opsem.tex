\begin{figure*}[!t]
\raggedright
%
\textbf{Syntax}\\
%
\begin{smathpar}
\renewcommand{\arraystretch}{1.2}
\begin{array}{lclcl}
\multicolumn{5}{c} {
  t \in \mathtt{Thread\; Ids} \qquad
  x,y \in \mathtt{Variables} \qquad
  c \in \mathtt{\{()\}} \cup \mathbb{N} \qquad
}\\
v & \in & \mathtt{Values} & \coloneqq & c \ALT \lambda x.\,s\\
s & \in & \mathtt{Expressions} & \coloneqq & v \ALT s\;s \ALT \run{s}{s}
   \ALT \fork{s} \ALT \pull \ALT \push{s}\\
p & \in & \mathtt{Programs} & \coloneqq & s_t \ALT p\,||\,p \\
f & \in & \mathtt{Tags} & \coloneqq & \C{INIT} \ALT \C{FORK} \;b 
  \ALT \C{PUSH} \ALT \C{MERGE} \;b\\
b & \in & \mathtt{Branches} & \coloneqq & [(v,f)] \ALT (v,f)::b \\
\end{array}
\end{smathpar}
%
\bigskip
%% If we are feeling adventurous, we can try defining e and s 
%% mutually recursively, such that their evaluation relations 
%% are also mutually recursive (multiple reduction steps of one 
%% relation is a single step of other). 

%
\textbf{Artifacts of Evaluation}\\
%
\begin{smathpar}
\renewcommand{\arraystretch}{1.2}
\begin{array}{lclcl}
H & \in & \mathtt{Branch\; Histories} & \coloneqq & t \mapsto b\\
E & \in & \mathtt{Eval.\; Contexts}(s) & \coloneqq & \bullet \ALT 
  \bullet\;s \ALT v\;\bullet \ALT \run{\bullet}{s}\\
P & \in & \mathtt{Eval.\; Contexts}(p) & \coloneqq & E_t \ALT 
  \bullet\,||\,p \ALT p\,||\,\bullet \\
\end{array}
\end{smathpar}
%
\bigskip

%
\textbf{Reduction Relation} \quad \fbox {$p;\;H \stepsto p';\;H'$} \\
%
%
\begin{smathpar}
  \begin{array}{lcll}
((\lambda x.s)\;v)_t;H & \stepsto & ([v/x]\,s)_t;H
            & [\rulelabel{E-App}]\\
(\run{v}{s})_t;\cdot & \stepsto & 
  s_t; \cdot[t_{\top} \mapsto [(v,\C{INIT})]]
            [t\mapsto [(v,\C{FORK}\; [(v,\C{INIT})])]] 
            & [\rulelabel{E-Run}]\\
(\fork{s})_t;H(t\mapsto (v,\_)::b) & \stepsto & 
    ()_t\,||\, s_{t'}; H[t'\mapsto [(v, \C{FORK} H(t))]] 
    \spc \texttt{where}\; t'\not\in dom(H)
            & [\rulelabel{E-Fork}]\\
(\push{v})_t;H & \stepsto & ()_t;H[t \mapsto (v,\C{PUSH})::H(t)]
            & [\rulelabel{E-Push}]\\
% & & & v\,=\,\C{merge}\,v\,v_1\,v_2 ~\texttt{and}~ \\
(\pull)_t;H(t \mapsto (v,\_)::m) & \stepsto & v_t;H
            & [\rulelabel{E-Pull}]\\
\end{array}
\end{smathpar}
%

% %
% \hspace*{\fill}[\rulelabel{E-Admin}]\hspace*{0.25in}
% \begin{smathpar}
% \begin{array}{c}
% \RULE
% {
%   s_t; H ~\stepsto^{*}~ v_t; H
% }
% {
%   E_t[s]; H ~\stepsto^{*}~ E_t[v]; H
% }
% \end{array}
% \end{smathpar}
% %

%
\hspace*{\fill}[\rulelabel{E-Pull-Wait}]
\begin{smathpar}
\begin{array}{c}
\RULE
{
  t\neq t' \spc
% \under{H}{v' \mbleto v} \spc
% \C{world}(H,t') \semsucceq \C{world}(H,t)\spc 
  v_m = \C{merge}(\C{lca}(H(t),H(t')), v, v') \spc
}
{
  (\pull)_t;H(t \mapsto (v,f)::m)(t' \mapsto (v',\_)::\_) ~\stepsto~
  (\pull)_t;H[t \mapsto (v_m,\C{MERGE}\; H(t'))::(v,f)::m]
}
\end{array}
\end{smathpar}
%

\caption{\name: Syntax and Operational Semantics}
\label{fig:opsem}
\end{figure*}
