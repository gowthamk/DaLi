\section{Introduction}

Distributed programs are challenging to write because they must juggle
two distinct mechanisms.  The first deals with application logic - how
do we define computation to be robust in the presence of disconnected
non-coherent communication among distributed threads of control?  The
second deals with system concerns - how do define notions of
atomicity, distribution, persistence, and replication?  Because of the
subtleties in dealing with the latter, especially when high
availability (low response latency) is desired, systems typically
separate these mechanisms into separate tools or services (e.g., CRDTs
have a well-defined semantics in an eventually consistent system that
may experience network partitions, while database and related storage
systems provide support for atomic transactions, persistence,
replication.  Unfortunately, because these different mechanisms have
vastly different semantics and goals, program structure becomes
complicated and muddled.  More importantly, it becomes problematic to
relate properties relevant to applications to the guarantees provided
by these lower-level services.  In this paper, we propose a radically
different vision of a distributed application that is predicated on
defining both mechanisms within the same language framework, with
those aspects not directly related to coherence and consistency
encapsulated within a library.  This not only simplifies program
reasoning, but leads to significant expressivity gains since
language-level types and abstractions are directly supported by the
library.  Realizing this vision leads to a radically different
programming model, one which realizes a monadic functional version
control system centered around data, rather than file coherence.  We
have implemented an end-to-end instantiation of this idea in OCaml
called DaLi and demonstrate how efficient distributed applications can
be written and deployed without having to deal with \emph{any}
system-centric features.

Points:
\begin{itemize}
  \item Language is often considered the pivot of the thought process.
  Unsurprisingly, programmers are conditioned to reason in terms of
  the language-level abstractions. However, reality of underlying
  systems often force programmers to think in terms of low-level
  abstractions that are counter-intuitive from the language
  standpoint. Even functional programmers are often exposed to this
  rather unfortunate state of affairs.

  \item Consider, for example, an OCaml programmer who wishes to
  manage an ordered set of elements in a binary tree, while using a
  hashtable to quickly obtain references into the tree. While the
  programmer enjoys high-level abstractions and safety guarantees of
  OCaml insofar as the program only accesses in-memory data, she has
  to let go of this convenience if the application requires data
  persistence. She has to now re-engineer her application around the
  data model enforced by the underlying storage engine (e.g.,
  relational data model, or key-value data model), complicating
  programming, and potentially compormising the language safety
  guarantees in the process. For instance, while OCaml guarantees the
  safety of all references, references between tables in a database
  are unsafe, and have been known to lead to bugs.

  \item Add replication to persistence, and the problems grow
  manifold. The impossibility of obtaining strong consistency
  guarantees without compromising scalability/availability under
  decentralized replication means that applications can now exhibit
  anomalous behaviors that are often incomprehensible to the
  programmer. The standard artifacts of replication made available to
  the programmer, namely mutable key-value pairs, expose the underlying
  system reality in its full ugliness. While the programmer may manage
  to reason about the contents of a key-value cell with much effort,
  the reasoningly unfortunately does not compose; each key-value pair
  can be manipulated independently, which makes it almost impossible
  to obtain guarantees over a collection of data items in the same
  vein as a high-level data structure (e.g., a binary search tree).
  CRDTs are an attempt to provide high-level semantics of replication
  for certain useful data types, however CRDTs do not come with a
  an associated replicated storage system and a programming framework,
  and programmers are again faced with the task of bridging the gap
  between the high-level semantics of the CRDT to the abstractions and
  guarantees offered by the underlying storage system. Moreover,
  standard data structures need significant re-engineering to be
  considered CRDTs (e.g., functions need to return \emph{effects} all
  of which need to commute), and programmers receive little or no tool
  support in this process. Moreover, CRDTs famously do not compose (in
  the sense that the invariants over the composition cannot be
  preserved), requiring programmers to resort to extraneous
  mechanisms with system-specific properties, such as transactions.

  \item In this paper we describe an OCaml library called \name that
  adresses the aforementioned problems. \name lets programmers add
  persistence and fully-decentralized replication to their
  applications without ever having to leave the comforts of high-level
  reasoning in OCaml. Thus, in keeping with the long tradition of
  Functional Programming, programmers are \emph{completely} shielded
  from the complexities of the underlying system model. This elevation
  from the low-level reasoning to the high-level reasoning concretely
  manifests as reasoning in terms of \emph{mergeable} data types
  instead of \emph{replicated} data types; replication being a
  system-level concern. A mergeable data type library is a standard
  OCaml data type library equipped with a \C{merge} function that
  describes how to merge two concurrent versions of the data type that
  originated from a single common ancestor. Thus, any OCaml data type
  for which a meaningful merge function can be written is a mergeable
  type. We give multiple examples of mergeable types useful in
  practice, present a simple theory that guides programmers in writing
  merge functions, and describe the tool support available to them in
  this process. \name also comes with a novel programming model that
  realizes a monadic functional version control system centered around
  data, rather than file coherence. The \name monad lets programmers
  write and compose concurrent computations around multiple versions
  of a mergeable data type. The computation makes progress by forking
  concurrent versions of the existing versions, creating later
  versions of the existing versions along a \emph{branch}, or by
  merging existing versions across the branches. However, as we shall
  see, unconstrained branching may yield a branching structure that
  becomes unmergeable, thus leading to \emph{stuck} states. An
  important property of the \name library is that it suitably
  constrains the branching structure to avoid stuck states without
  introducing additional coordination that compromises the
  decentralized nature of the computation. Futhermore, \name
o  profitably exploits the provenance information available via the
  branching structure to make the application resilient to network
  partitions. As a result, OCaml programmers get the benefits of
  persistence and highly available replication, while they continue to
  enjoy the comforts of high-level reasoning and the familiarity of
  the standard library data types.

  Our contributions are summarized below:
  \begin{itemize}
    \item We formally introduce the concept of a mergeable data type,
    as an alternative to a replicated data type, to admit high-level
    reasoning about replicated data in functional programs.

    \item We describe the \name library that transparently adds
    persistence and replication features to a mergeable type. \name
    comes with a programming model that brings to bear the power and
    flexibility of a version control system on administration of the
    replicated data. \name however hides the complexity of version
    control behind a monad, and exposes its functionality via a simple
    API that lets programmers define and compose concurrent OCaml
    computations around replicated data. 

    \item We present a formalization of the \name programming model,
    and prove results that guarantee eventual convergence of
    concurrent versions, and progress (stuck-freeness) guarantees.

    \item We describe an implementation of \name atop the Irmin store,
    and describe multiple examples, and two case studies that
    attest to the practical utility of \name.
  \end{itemize}

\end{itemize}
