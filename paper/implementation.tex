\section{Implementation}
\label{sec:implementation}

We describe the realization of \name programming model on top of a persistent
distributed store and the language extensions to enable and enforce the
constraints of the \name.

\name programming model is realized on top of Irmin~\cite{irmin}, an OCaml
library database implementation that is part of the MirageOS
project~\cite{mirage}. Irmin provides a persistent multi-versioned store with
with a content-addressable heap abstraction. Simply put, content-addressability
means that the address of a data block is determined by its content. If the
content changes, then so does the address. Old content continues to be
available the old address. Content-addressability also results in constant time
structural equality checks, which we exploit in our mergeable rope
implementation (Section~\ref{sec:ropes)}.

Irmin provides support for distribution, fault-tolerance and concurrency
control by incorporating Git distributed version control~\cite{git} protocol
over its object model. Indeed, Irmin is fully compatible with the Git command
line tools. Distributed replicas in \name are created by cloning a \name
repository. Due to \name's mergeable type, each replica can operate completely
independently, accepting client requests, even when disconnected from other
replicas, resulting in a fully available distributed database system.

Concurrent operations in Irmin are tracked within the notion of branches,
allowing the programmer to explicitly merge the branches on demand. \name
realizes the concurrent programming model on top of the notion of branches;
each replica operates on its own branch, and thus is isolated from actions on
other replicas. \name monad conceals the branching structure, but also
transparently performs the necessary merges to obtain a history graph with a
unique lowest common ancestor.

Importantly, Irmin supports user-defined three-way merge functions for
reconciling concurrent operations. While Irmin's merge functions are defined
over objects on Irmin's content-addressable heap, \name's merge functions are
defined over OCaml types. We address this representational mis-match with the
help of OCaml's PPX metaprogramming support to derive bi-directional
transformations between objects on OCaml and Irmin heaps. We also derive the
various serialization functions required Irmin. This address the representation
mismatch between the application and storage layers. Synchronization between
replicas is performed using Git's notion of pushing and pulling updates from
remotes over the Git transfer protocol~\cite{git-tp}. As a result, we reap the
benefits of efficient delta-transfer (only missing objects are transferred
between replicas), compression and end-to-end encrypted communication between
the replicas.
