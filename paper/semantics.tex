\section{Formalization}
\label{sec:formalization}

\subsection{Operational Semantics}
\label{sec:opsem}

We formalize our ideas in the context of a lambda calculus ($\lang$)
shown in Fig.~\ref{fig:opsem}. Expressions of $\lang$ are variables,
constants, and \name primitives composed using the lambda combinator.
For brevity, we use short names for \name primitives: \C{run} for
\C{with\_init\_version\_do}, and \C{fork} for \C{fork\_version}. To
simplify the technical development, \name's \C{sync\_next\_version}
operation is broken down into two primitives - \C{push} and \C{pull},
which can be composed to get the desired effect of the original:
$\C{sync}\;x \;=\; (\lambda y.\pull)\; (\push\,x)$.  The semantics
of\C{get\_current\_version} is subsumed by \C{pull}, hence elided.
Values ($v$) are constants and lambda abstractions.  A program ($p$)
is a parallel composition of threads, where each thread is an
expression ($s$) indexed by the corresponding thread identifier ($t$).

\begin{figure*}[!t]
\raggedright
%
\textbf{Syntax}\\
%
\begin{smathpar}
\renewcommand{\arraystretch}{1.2}
\begin{array}{lclcl}
\multicolumn{5}{c} {
  t \in \mathtt{Thread\; Ids} \qquad
  x,y \in \mathtt{Variables} \qquad
  c \in \mathtt{\{()\}} \cup \mathbb{N} \qquad
}\\
v & \in & \mathtt{Values} & \coloneqq & c \ALT \lambda x.\,s\\
s & \in & \mathtt{Expressions} & \coloneqq & v \ALT s\;s \ALT \run{s}{s}
   \ALT \fork{s} \ALT \pull \ALT \push{s}\\
p & \in & \mathtt{Programs} & \coloneqq & s_t \ALT p\,||\,p \\
f & \in & \mathtt{Tags} & \coloneqq & \C{INIT} \ALT \C{FORK} \;b 
  \ALT \C{PUSH} \ALT \C{MERGE} \;b\\
b & \in & \mathtt{Branches} & \coloneqq & [(v,f)] \ALT (v,f)::b \\
\end{array}
\end{smathpar}
%
\bigskip
%% If we are feeling adventurous, we can try defining e and s 
%% mutually recursively, such that their evaluation relations 
%% are also mutually recursive (multiple reduction steps of one 
%% relation is a single step of other). 

%
\textbf{Artifacts of Evaluation}\\
%
\begin{smathpar}
\renewcommand{\arraystretch}{1.2}
\begin{array}{lclcl}
H & \in & \mathtt{Branch\; Histories} & \coloneqq & t \mapsto b\\
E & \in & \mathtt{Eval.\; Contexts}(s) & \coloneqq & \bullet \ALT 
  \bullet\;s \ALT v\;\bullet \ALT \run{\bullet}{s}\\
P & \in & \mathtt{Eval.\; Contexts}(p) & \coloneqq & E_t \ALT 
  \bullet\,||\,p \ALT p\,||\,\bullet \\
\end{array}
\end{smathpar}
%
\bigskip

%
\textbf{Reduction Relation} \quad \fbox {$p;\;H \stepsto p';\;H'$} \\
%
%
\begin{smathpar}
\begin{array}{lcll}
(\run{v}{s})_t;\cdot & \stepsto & 
  s_t; \cdot[t_{\top} \mapsto [(v,\C{INIT})]]
            [t\mapsto [(v,\C{FORK}\; [(v,\C{INIT})])]] 
            & [\rulelabel{E-Run}]\\
(\fork{s})_t;H(t\mapsto (v,\_)::b) & \stepsto & 
    ()_t\,||\, s_{t'}; H[t'\mapsto [(v, \C{FORK} H(t))]] 
    \spc \texttt{where}\; t'\not\in dom(H)
            & [\rulelabel{E-Fork}]\\
(\push{v})_t;H & \stepsto & ()_t;H[t \mapsto (v,\C{PUSH})::H(t)]
            & [\rulelabel{E-Push}]\\
% & & & v\,=\,\C{merge}\,v\,v_1\,v_2 ~\texttt{and}~ \\
((\lambda x.s)\;v)_t;H & \stepsto & ([v/x]\,s)_t;H
            & [\rulelabel{E-App}]\\
(\pull)_t;H(t \mapsto (v,\_)::m) & \stepsto & v_t;H
            & [\rulelabel{E-Pull}]\\
\end{array}
\end{smathpar}
%

% %
% \hspace*{\fill}[\rulelabel{E-Admin}]\hspace*{0.25in}
% \begin{smathpar}
% \begin{array}{c}
% \RULE
% {
%   s_t; H ~\stepsto^{*}~ v_t; H
% }
% {
%   E_t[s]; H ~\stepsto^{*}~ E_t[v]; H
% }
% \end{array}
% \end{smathpar}
% %

%
\hspace*{\fill}[\rulelabel{E-Pull-Wait}]
\begin{smathpar}
\begin{array}{c}
\RULE
{
  t\neq t' \spc
  \under{H}{v' \mbleto v} \spc
% \C{world}(H,t') \semsucceq \C{world}(H,t)\spc 
  v_m = \C{merge}(\C{lca}(H(t),H(t')), v, v') \spc
}
{
  (\pull)_t;H(t \mapsto (v,f)::m)(t' \mapsto (v',\_)::\_) ~\stepsto~
  (\pull)_t;H[t \mapsto (v_m,\C{MERGE}\; H(t'))::(v,f)::m]
}
\end{array}
\end{smathpar}
%

\caption{\name: Syntax and Operational Semantics}
\label{fig:opsem}
\end{figure*}


Fig.~\ref{fig:opsem} also shows the syntax of \emph{branches}, which
are artifacts of evaluation and only appear during run-time. A branch
is a non-empty sequence of tagged values, where the tag captures the
abstract run-time operation that led to the creation of the value. It
is implicitly assumed that each value added to a branch is uniquely
identifiable, and hence no two values on a branch are equal.  The
uniqueness assumption is later extended to a collection of branches
that constitute a branching structure. A real implementation meets
this assumption by versioning values across the branches. Thus, in
reality, branches contain \emph{versions} which denote values.  To
simplify the presentation, the semantics, does not make this
distinction, and uses values and versions interchangeably.

The small-step operational semantics of $\lang$ is defined via a
reduction relation ($\stepsto$) that relates \emph{program states}. A
program state ($p;\,H$) consists of a program $p$ and a \emph{branch
history} $H$ that maps thread identifiers to corresponding branches.
Each thread is associated with a branch during evaluation. Evaluation
contexts have been defined separately for expressions ($E$) and
programs ($P$), with the latter subsuming the former.

$E$ is defined to evaluate the first argument of a \C{run} expression
to a value that constitutes an initial version (recall that \C{run}
models \nameMonad's \C{with\_init\_version\_do}). A program evaluation
context non-deterministically picks an extant thread to evaluate. The
administrative rule that relates transitions of holes to transitions
of expressions and programs is straightforward, and hence elided. The
remaining core reduction rules are presented in
Fig.~\ref{fig:opsem}. For brevity, we write $H(t\mapsto (v,f))$ to
denote the proposition that $H$ maps $t$ to $(v,f)$. The notation $H[t
  \mapsto (v,f)]$ denotes the extension of $H$ with the binding $t
\mapsto (v,f)$, as usual.

Reduction rules let expression evaluation take a step by rewriting the
expression and suitably updating the branch history ($H$).
\rulelabel{E-App} is the standard beta reduction rule.  The
\rulelabel{E-Run} rule is applicable only when $H$ is empty, i.e.,
when no prior branching structure exists, and commences a distributed
computation with a corresponding version tree. The rule rewrites the
$\C{run}\;v\;s$ expression to $s$, while creating a new branching
structure with two branches: a \emph{top} branch that has just the
initial version (tagged with \C{INIT}), and a branch for the current
thread ($t$) forked-off from the top branch.  The first version on the
current branch ($H(t)$) denotes the same value ($v$) as the initial
version on the top branch, although versions themselves are deemed
distinct. The new version is tagged with a \C{FORK} tag that keeps the
record of its origin, namely the \C{fork} operation and the branch from
which the current branch is forked. The \rulelabel{E-Fork} rule forks
a new thread with a fresh id ($t'$) and adds it to the thread pool.
The corresponding branch ($H(t')$) is forked from the parent thread's
branch ($H(t)$). The semantics of branch forking is the same as
described above. The \C{fork} expression in the parent thread
evaluates to \C{()}. The \rulelabel{E-Push} rule creates a new version
on the current branch ($H(t)$) using the pushed value ($v$).  Although
our semantics does not directly expose heap-allocated values, the
intention is that $v$ is a replicated object, manipulated on the local
heap that, after push, now becomes subject to merging and coordination
with other replicas.

The semantics non-deterministically chooses \rulelabel{E-Pull} or
\rulelabel{E-Pull-Wait} rules to reduce a \C{pull} expression. The
\rulelabel{E-Pull} rule reduces \C{pull} to \C{()}, and returns the
latest version on the current branch. The \rulelabel{E-Pull-Wait} rule
can be thought of as a stutter step; it doesn't reduce \C{pull}, but
updates the branching structure by merging (the latest version of) a
concurrent branch ($H(t')$) into (the latest version of) the current
branch ($H(t)$), and extending the current branch with the merged
version ($v_m$). The versions are merged only if they are \emph{safely
mergeable} (denoted $\under{H}{v' \mbleto v}$) - a notion that we will
explain shorty. The new version is tagged with a \C{MERGE} tag that,
like a \C{FORK} tag, records its origin. The rule assumes a function
\C{lca} that computes the \emph{lowest common ancestor} (LCA) for the
latest versions on the given pair of branches. The formal definition
of LCA is given below. The \rulelabel{E-Pull-Wait} and
\rulelabel{E-Pull} rules thus let a thread synchronize with other
distributed threads manipulating versions of replicas in multiple
steps before returning the result of the \C{pull}. Since \C{sync} is a
composition of \C{push} and \C{pull}, its behavior can be explained
thus: \C{sync} pushes the given value onto the current (local) branch,
merges a (possibly empty) subset of concurrent branches into the local
branch, and returns the result.  (The operation is not guaranteed to
synchronize with all concurrent branches because not all such branches
may be available, because of e.g., network partitions.)

To define LCA, we formalize the intuitive notation of the ancestor
relationship between versions of a legal branching history (i.e., a
branching history generated by the rules in Fig.~\ref{fig:opsem}):

\begin{definition} [\bfseries Ancestor]
Version $v_1$ is a ancestor of version $v_2$ under a history
$H$ (written $\under{H}{v_1 \preceq v_2}$) if and only if one of the
following is true:
\begin{itemize}
  \item There exists a branch $b$ in $H$ (i.e., $\exists(t\in
  dom(H)).\,H(t) = b$) in which $v_2$ immediately succeeds
  $v_1$,
  \item There exists a branch $b$ in $H$ that contains $(v_2,
  \C{FORK}\; (v_1,f_1)::b_1)$, for some $f_1$ and $b_1$,
  \item There exists a branch $b$ in $H$ that contains
  $(v_2, \C{MERGE}\;(v_1,f_1)::b_1)$, for some $f_1$ and $b_1$,
  \item $v_1 = v_2$, or $v_1$ is transitively an ancestor of
  $v_2$, i.e., $\exists v.~ \under{H}{v_1 \preceq v} \conj
  \under{H}{v \preceq v}$
\end{itemize}
\end{definition}

The ancestor relation is therefore a partial order with a greatest
lower bound (the initial version).  Thus, for any two versions in a
legal history, there exists at least one common ancestor.  The ancestor
relationship among all common ancestors lets us define the notion of a
lowest common ancestor (LCA):

\begin{definition} [\bfseries Lowest Common Ancestor]
Version $v$ is said to be a common ancestor of versions $v_1$ and
$v_2$ under a history $H$ if and only if $\under{H}{v \preceq v_1}$
and $\under{H}{v \preceq v_2}$. It is said to be the lowest common
ancestor (LCA) of $v_1$ and $v_2$, iff there does not exist a $v'$
such that $\under{H}{v' \preceq v_1}$ and $\under{H}{v' \preceq v_2}$
and $\under{H}{v \preceq v'}$.
\end{definition}

The function \C{lca} used in the premise of \rulelabel{E-Pull-Wait}
computes the LCA of two branches, and assumes it to be unique.  But,
observe that the definition of LCA doesn't guarantee its uniqueness;
it has to be enforced explicitly.  The semantics enforces the
uniqueness of LCA by constraining the branching structure via the
\rulelabel{E-Pull-Wait} rule. The rule merges version $v'$ into
version $v$ only if they are safely mergeable ($\under{H}{v' \mbleto
  v}$), i.e., they have a single LCA, and merging both versions does
not lead to a case where a branch (i.e., its latest version) has
multiple LCAs with remaining branches, and hence gets
\emph{stuck}. Such nuances of LCA, and the guarantees provided by the
system notwithstanding these nuances, are discussed in 
Sec.~\ref{sec:meta}.

\subsection{Mergeable Data Type}

\begin{figure}[!t]
\centering
\subcaptionbox[] {\small
  When a thread \C{pushes} a new version, it is presumably a semantic
  successor of the version it last \C{pulled} into the heap.
  \label{fig:syntactic-ancestor-1}
} [0.47\columnwidth] {
  \includegraphics[scale=0.7]{Figures/semantic-ancestor-1}
}
\hfill
\subcaptionbox[] {\small
  Versions created by the \C{merge} operation are syntactic successors
  of merged versions, but need not necessarily be semantic successors.
  \label{fig:syntactic-ancestor-2}
} [0.47\columnwidth] {
  \includegraphics[scale=0.8]{Figures/semantic-ancestor-2}
}
\caption{New versions are created from existing versions either
through \C{push} or \C{merge}.}
\label{fig:syntactic-ancestors}
\end{figure}

We now formally define a mergeable type.  A functional data type
library is a tuple consisting of a type ($t$), and a collection of
functions ($f$, $g$, etc.) of type $t \rightarrow t$ that are
\emph{primitive morphisms}. A morphism ($P$, $Q$, etc.) is either a
primitive morphism, or an associative composition of morphisms ($P
\circ Q \circ R$). An object $B:t$ is called a \emph{semantic
successor} of $A:t$ (conversely, $A$ is called a \emph{semantic
ancestor} of $B$) if and only if there exists a morphism $P$ such that
$P(A) = B$.

The aim of a three-way merge function over a type $t$ is to merge a
pair of semantic successors, $B:t$ and $C:t$, of an object $A:t$, into
another object $D:t$ such that the relationship between the semantic
successors and $D$ satisfies certain conditions. These conditions can
be understood by observing the \rulelabel{E-Pull-Wait} rule of
Fig.~\ref{fig:opsem}, which applies the \C{merge} function to a pair
of concurrent versions ($v_1$ and $v_2$) and their least common
ancestor ($v$). Thus, the only relationship that exists between $v$
and $v_1$, and $v$ and $v_2$ is the syntactic ancestor relationship
that follows from the computation's branching structure. The merge
function, as described above, however assumes that concurrent versions
are semantic successors of their LCA.  It is therefore essential to
maintain coherence between the syntactic and semantic ancestor
relations.

The ways in which syntactic ancestor relationships are created among
versions is captured in Fig.~\ref{fig:syntactic-ancestors}. Whenever an
object is pushed onto the branch, an ancestor relationship is created
between the previous version $v_1$, and the newly created version
$v_2$ (Fig.~\ref{fig:syntactic-ancestor-1}). However, since $v_2$ is
pushed by the thread after reading $v_1$ into the heap, it is
reasonable to assume that $v_2$ is a result of applying a morphism $P$
to $v_1$ (i.e., $\exists P.~v_2 = P(v_1)$).  Hence, $v_1$ is a
semantic ancestor of $v_2$.  Fig.~\ref{fig:syntactic-ancestor-2}
captures another way of establishing an ancestor relationship, namely by
merging branches. Version $v_{21}$ on branch $b_2$ is merged into
$v_{11}$ on $b_1$ to create $v_{12}$ on $b_1$. Versions $v_{11}$ and
\begin{wrapfigure}{l}{.4\textwidth}
\centering
\includegraphics[scale=0.35]{Figures/pushouts}
\caption{Commutative diagram representing the merge operation.}
\label{fig:pushouts}
\end{wrapfigure}
$v_{21}$ are now syntactic ancestors of $v_{12}$, but for them to be
semantic ancestors, the merge function has to enforce the relationship
explicitly. In other words, the result of merging  a pair of
semantic successors, $B$ and $C$, of an $A$, has to be an object $D$
that is a semantic successor of both $B$ and $C$. We now formalize
this intuition to define a mergeable type.
\begin{definition} [\bfseries Mergeable Type]
\label{def:mergeable-type}
A type $t$ is said to be mergeable, if and only if there exists a
function $M$ of type $t \times t \times t \rightarrow (t \rightarrow
t)\times(t \rightarrow t)$ that satisfies the following property:
$\forall (A, B, C\,:\,t).~ (\exists (P,Q\,:\, t \rightarrow t).~ B =
P(A) \conj C = Q(A)) \Rightarrow (\exists (P',Q': t\rightarrow
t).~M(A,B,C) = (P',Q') \conj  Q'(B) = P'(C))$
\end{definition}
Intuitively, a type $t$ is a mergeable type if, whenever there exist
two morphisms $P$ and $Q$ that map $A:t$ to $B:t$ and $C:t$, there
also exists morphisms $P'$ and $Q'$, that map both $B$ and $C$ to
$D:t$. This intuition is expressed visually in
Fig.~\ref{fig:pushouts}.  For a mergeable type $t$, we have the
guarantee that a sequence of values read by a thread from its branch
is sensible, as per the data type semantics:

\begin{theorem} [\bfseries Branch-local consistency]
\label{thm:branch-consistency}
Let $t$ be a mergeable type, and let $H$ be a legal branching history
over values of type $t$. For every pair of values $v_1:t$ and $v_2:t$
along a branch $b$ in $H$, if $\under{H}{v_1 \preceq v_2}$, then there
exists a morphism $P:t\rightarrow t$ such that $v_2 = P(v_1)$.
\end{theorem}

Rather than specifying merge ($M$) in terms of an LCA,
Definition~\ref{def:mergeable-type} specifies $M$ in terms of a pair
of morphisms.  This specification is useful to avoid the expression of certain
nonsensical merge functions, and thus enforce a useful branch-local
consistency property (Theorem~\ref{thm:branch-consistency}). For
instance, consider the counter data type from Sec.~\ref{sec:intro}. A
three-way merge function can be defined to return a constant
regardless of its arguments:
\begin{ocaml}
  let merge lca a b = 0
\end{ocaml}
Such a merge function allows the client to
witness the violation of the branch-local consistency property. For
example, the client may \C{sync} the counter, whose current local
value is $10$, and obtain $0$ as a result, thus witnessing a violation
of counter's monotonicity property. Fortunately,
Def.~\ref{def:mergeable-type} disallows this semantics. It is
impossible to define the function $M$ for the counter data type that
violates the monotonicity invariant, because no counter morphism
violates the invariant. In general, the specification of $M$
guarantees that the merge operation for a type $t$ preserves any
invariants (e.g., balancedness of a tree, sortedness of a list,
non-negativeness of an integer etc) that are preserved by all of $t$'s
morphisms.  The commutativity of the diagram in
Fig.~\ref{fig:pushouts} need not be enforced statically. It can
equally be verified at run-time by checking that the morphisms returned by $M$
map respective concurrent versions to the same value. The
\rulelabel{E-Pull-Wait} rule of the operational semantics
(Fig.~\ref{fig:opsem}) can be extended to this effect:
\begin{smathpar}
\begin{array}{c}
\RULE
{
  t\neq t' \spc
  \under{H}{v' \mbleto v} \spc
% \C{world}(H,t') \semsucceq \C{world}(H,t)\spc 
  v' \not\preceq v \spc
  (P',Q') = M(\C{lca}(H(t),H(t')), v, v') \spc
  Q'(v) = P'(v')
}
{
  (\pull)_t;H(t \mapsto (v,f)::m)(t' \mapsto (v',\_)::\_) ~\stepsto~
  (\pull)_t;H[t \mapsto (v_m,\C{MERGE}\; H(t'))::(v,f)::m]
}
\end{array}
\end{smathpar}

% While the merge function $M$ as specified by
% Def.~\ref{def:mergeable-type} has more verification value, we continue
% to use the three-way merge function \C{merge} in our examples for
% flexibility. 

Def.~\ref{def:mergeable-type} is immediately applicable to simple data
types like counter to explain why they are mergeable. The merge logic
for counter is simple enough that it does not require great effort to
identify the $P$ and $Q$ that lead to the concurrent counter versions,
and transform them into $P'$ and $Q'$ such that the diagram in
Fig.~\ref{fig:pushouts} commutes. For more
\begin{wrapfigure}{l}{.4\textwidth}
\centering
\includegraphics[scale=0.35]{Figures/pushouts-2}
\caption{Commutative diagram for Def.~\ref{def:mergeable-type-2}.}
\label{fig:pushouts-2}
\end{wrapfigure}
sophisticated data types, such as list, this process is more involved,
and may require the programmer to perform non-trivial reasoning. We
now present an alternative definition of a mergeable type that is
stronger than Def.~\ref{def:mergeable-type}, but also serves as a
recipe to build non-trivial merge functions, like that of a list.

\paragraph{Notation} As mentioned earlier, we distinguish between the
named functions defined by a data type library (primitive morphisms),
and their compositions (morphisms). A sequence of primitive morphisms
is written $P^*$ or $Q^*$. The composition of a sequence of primitive
morphisms (written $\denot{P^*}$) is their right-to-left composition,
i.e., $\C{fold\_left}\;(\lambda f.\lambda g. f \circ g)\; id\; P^*$,
where $\C{fold\_left}$ has its usual OCaml type. The length of a
sequence $S$ is written $\length{S}$.

\begin{definition} [\bfseries Mergeable Type]
\label{def:mergeable-type-2}
A type $t$ is said to be mergeable if and only if the following
conditions hold:
\begin{itemize}
  \item There exists a function $W: t \times t \rightarrow (t
  \rightarrow t)^*$ that accepts an object $A:t$ and its
  semantic successor $B:t$, and returns a minimal sequence of
  primitive morphisms $P^*$, whose composition maps $A$ to $B$. That
  is, $W(A,B) = P^*$, where $\denot{P^*}(A) = B$, and there does not
  exist an $R^*$ such that $\denot{R^*}(A) = B$ and $\length{R^*} <
  \length{P^*}$

  \item There exists a function $M: (t \rightarrow t)^*\!\times\!(t
  \rightarrow t)^* \;\rightarrow\; (t \rightarrow t)\!\times\!(t
  \rightarrow t)$ that accepts a pair of minimal sequences of
  primitive morphisms, $P^*$ and $Q^*$, and returns a pair of morphisms,
  $P'$ and $Q'$, such that for any object $A:t$, $(Q' \circ
  \denot{P^*})(A) = (P' \circ \denot{Q^*})(A)$.  
\end{itemize}
The Commutativity diagram of Fig.~\ref{fig:pushouts-2} visualizes this
definition. 
\end{definition}

The definition asserts the existence of two functions for a mergeable
type $t$. First, the function $W$, which (implicitly) computes the
edit distance between an object $A:t$ and its semantic successor
$B:t$.  The sequence of primitive morphisms it returns ($P^*$) is
called an edit sequence. Applying the edit sequence on $A$ produces
$B$ (i.e., $\denot{P^*}(A) = B$). Second, the definition asserts the
existence of a function $M$, which performs an operational
transformation of the primitive morphisms in an edit script. An
operational transformation of a primitive morphism $P$ w.r.t a
morphism $Q$ is another primitive morphism $P'$ such that for any
object $A:t$, $P'$ has the same \emph{effect} on $Q(A)$ as $P$ has on
$A$.

Def.~\ref{def:mergeable-type-2} is immediately applicable to the list
type, and lets us explain why it is a mergeable type.  \C{MList.edit\_seq}
and \C{MList.op\_transform}, respectively, are the evidences for the
existence of $W$ and $M$ functions.
  



\subsection{Properties of the LCA}
\label{sec:meta}

When merging two concurrent versions $v_1$ and $v_2$, the common
ancestor argument for \C{merge} must be the LCA of $v_1$ and $v_2$,
without which \C{merge} may yield unexpected results. This is
demonstrated for the grow-only counter in
Fig.~\ref{fig:merge-needs-lca}, where an incorrect count is obtained if a
common ancestor that is not an LCA is used to merge 4 and 7. While in
this example there is a unique LCA for 4 and 7, in general this may
\begin{wrapfigure}{l}{.4\textwidth}
\centering
\includegraphics[scale=0.6]{Figures/merge-needs-lca}
\caption{This example of a grow-only counter illustrate why \C{merge}
needs a lowest common ancestor, and not just a common ancestor. Both 0
and 2 are common ancestors of 4 and 7, while 2 is their lowest common
ancestor (since $0 \preceq 2$). The result (v) of merging 4 and 7 is
11 (incorrect) if 0 is used as the common ancestor for merge, and 9
(correct, because 2+2+3+2 = 9) if 2 is used. }
\label{fig:merge-needs-lca}
\end{wrapfigure}
not be the case. With unrestrained branching and merging, there is no
bound on the number of LCAs a pair of versions can have.  For example,
in Fig.~\ref{fig:criss-cross-lcas}, the merge of 0 with 3 is preceded
by two ``criss-cross'' merges between their respective
branches\footnote{
  When discussing merges and LCAs, we often attribute the properties
  of latest versions on branches to the branches themselves.  For
  instance, when we say two branches merge, in fact their latest
  versions merge. Likewise, LCA of two branches means the LCA of their
  latest versions.
}
resulting in there being two LCAs (5 and 4) for 0 and 3.
Multiple LCAs can occur even without criss-cross merges, as
demonstrated by Fig.~\ref{fig:external-lcas}.
Concurrent versions with multiple LCAs do not lend themselves to
three-way merging. If such versions are latest on their respective
branches, they render the branches unmergeable (since  \C{lca} is no
longer a function) as demonstrated by examples in
Fig.~\ref{fig:many-lcas}. Note that for both the examples in
Fig.~\ref{fig:many-lcas}, no extension of the branching structure
can make the branches merge again. Thus the system is effectively
\emph{partitioned} permanently. This is clearly a problem.

\begin{figure}[!t]
\centering
\subcaptionbox[] {\small
  In this example, 1 and 3 have two LCAs (3 and 4) a result of
  previous merges. The dotted circle denotes a virtual ancestor
  obtained by merging the two LCAs.
  \label{fig:criss-cross-lcas}
} [0.47\columnwidth] {
  \includegraphics[scale=0.55]{Figures/2-LCAs}
}
\hfill
\subcaptionbox[] {\small
  In this example, latest versions on $b_1$ and $b_3$, and
  $b_2$ and $b_3$ have two LCAs each, hence are unmergeable. 
  \label{fig:external-lcas}
} [0.47\columnwidth] {
  \includegraphics[scale=0.7]{Figures/2-external-LCAs}
}
\caption{Examples where two branches have more than one LCA, hence
cannot merge. }
\label{fig:many-lcas}
\end{figure}

The problem of multiple LCAs also arises in the context of source
control systems, which employ \emph{ad hoc} mechanisms to pave the way
for three-way merging.  Git~\cite{git}, for instance,
recursively merges LCAs by default to compute a virtual ancestor, which then
serves as the LCA for merging concurrent versions. This method is
demonstrated in the branching structure for a mergeable, replicated
counter as shown in Fig.~\ref{fig:criss-cross-lcas}, where LCAs 5 and
4 of 0 and 3 are merged (with their LCA being 10) to generate -1 as
the virtual LCA to merge 5 and 4. A major downside with this approach
is that it makes no guarantees on the relationship between the virtual
ancestor and its concurrent versions; the former may not even be a
legal ancestor of the latter as per the semantics of the data type.
For instance, suppose the integer type in
Fig.~\ref{fig:criss-cross-lcas} represents a bank account balance,
which is expected to disallow any activity on the account if the
balance is ever known to be less than zero.  From the perspective of
the library designer and its clients, there is no meaningful scenario
in which versions 3 and 0 can emerge from -1, since the only
transition allowed by the semantics from -1 is to itself.  Clearly,
\emph{ad hoc} mechanisms like this are error-prone and difficult to
apply in general.

Fortunately, unlike source control systems where branching structure
is entirely dictated by the user, \name abstracts away branching
structure from the programmer, and hence retains the ability to
manifest it in a way that it deems fit. In particular, \name solves
the problem of multiple LCAs by suitably constraining the branching
structure such that the problem never arises. The constraints are
imposed either implicitly, as a result of how the operational semantics
defines an atomic step, or explicitly, by insisting that certain
conditions be met before merging a pair of versions
(\rulelabel{E-Pull-Wait}). First, the operational semantics already
disables criss-cross merges since it only ever merges versions that
are latest on their respective branches. Second, we impose certain
pre-conditions on the merging branches to preempt the structure shown
in Fig.~\ref{fig:external-lcas}. The intuition is as follows: consider
the branch $b_3$ at the instance of merging $v_{12}$. Since it has
already merged $v_{22}$, $v_{22}$ could be a common ancestor for $b_3$
and some other branch (call it $b$). Now, if $b_3$ merges $v_{12}$,
the same could be true of $v_{12}$. Since $v_{12}$ and $v_{22}$ are not
orderd by the ancestor relation, both become LCAs of
$b_3$ and $b$. We observe that this scenario can be prevented if, when
merging $b_1$, $b_3$ insists on an ancestor relation between the last
merged version ($v_{22}$) and the currently merging version.  We call
the last merged version of a branch its \emph{locus}. By maintaining a
total order of among locii of a branch $b$ as new versions are
created, we effectively guarantee the presence of a version (the
locus) that is lower than all the ancestors of the latest version on
$b$. Next, by requiring the locus to be an ancestor of the merging
version, we ensure that all the common ancestors have a single lowest
element, thus enforcing the uniqueness of the LCA.

We now formalize the intuitions described above via a series of
definitions that help us state the guarantees offered by the system.

\begin{definition} [\bfseries Internal and External Ancestors]
Given a branch $b$ and a version $v\in b$, an internal ancestor
($\preceq_i$) of $v$ is an ancestor from the same branch $b$. An
external ancestor ($\preceq_o$) of $v$ is an ancestor from a different
branch $b'\neq b$.
\end{definition}

\begin{definition} [\bfseries Locus]
Given a branch $b$ and a version $v\in b$, the locus ($v_o$) of $v$ is an
ancestor that is not an ancestor of any other external ancestor of
$v$. That is, $\under{H}{v_o \preceq v}$, and there does not exist a
$v_o' \not\in b$ such that $\under{H}{v_o' \preceq v}$ and
$\under{H}{v_o \preceq v_o'}$. We lift the notion of locus to the
level of branches by defining the locus of a branch as the locus of
its latest version.
\end{definition}

In a legal branching history, every version has a unique locus
(property follows from Def.~\ref{def:mergeability} below). We
define the \emph{causal history} of a version as the set of all
ancestors of that version. A locus ($v_o$) of a version ($v$) is
therefore the maximum element (lowest element in the ancestor
relation) of the causal history of $v$'s internal ancestor that was
the result of the last merge. We make use of this observation while
defining mergeability.

\begin{definition} [\bfseries Mergeability]
\label{def:mergeability}
Given a history $H$, a version $v_1$ and a version $v_2$ that is not
an ancestor of $v_1$ under $H$, $v_2$ is mergeable into $v_1$ (denoted
$\under{H}{v_2 \mbleto v_1}$) if and only if the locus ($v_o$) of
$v_1$ is an ancestor of $v_2$, and no internal ancestor of $v_1$ that
is not also an ancestor of $v_o$ is an ancestor of $v_2$.
\end{definition}

If $v_1$ and $v_2$ referred in the above definition are the latest
versions on their respective branches $b_1$ and $b_2$, we say $b_2$ is
mergeable into $b_1$, or more generally, $b_1$ and $b_2$ are
mergeable. The definition essentially requires all of $v_1$'s causal
history that preceeds its locus $v_o$ to be included in $v_2$'s causal
history, and none of $v_1$'s history that succeeds $v_o$ to be
included in $v_2$'s history. This allows us to view versions $v_1$ and
$v_2$ as having been independently evolved from a common causal
history whose maximum (w.r.t the ancestor relation) is the version
$v_o$. Consquently, $v_o$ becomes the LCA for $v_2$'s merge into
$v_1$. Generalizing this observation for the latest versions on
any two branches, we state the following theorem:

\begin{theorem} [\bfseries Unique LCA]
Every pair of branches in a legal history $H$, if mergeable as per
Def.~\ref{def:mergeability}, have a unique least common ancestor.
\end{theorem}

While the definition of mergeability is sufficient to enforce
uniqueness of LCAs, it may hinder progress in the sense that there may
not exist a pair of branches in a legal history satisfying the
definition, thus making the progress impossible. The following theorem
asserts that this is not the case:

\begin{theorem} [\bfseries Progress]
In a legal branching history $H$ produced by the operational
semantics, if two branches, $b_i$ and $b_j$ are not mergeable (as per
Def.~\ref{def:mergeability}), then there exists a sequence of fork and
merge operations (between mergeable versions) that can be performed on
$H$ to yeild a new history $H'$, where $b_i$ and $b_j$ are mergeable.
\end{theorem}

The proof is by the induction on the size of branching history. The
key observation is that the smallest history that is the union of
causal histories of two unmergeable versions (call them $v_i$ and
$v_j$) is smaller than the history that includes the unmergeable
versions. Since causal histories have maximum versions (respective
locii), which, by the inductive hypothesis, can be made mergeable, we
merge them to yield a new version $v$ that is the maximum of the
causal histories of both $v_i$ and $v_j$. If $v_i$ and $v_j$ are the
latest versions on their respective branches, the branches are now
mergeable.

% Observe that $v$ and $v_i$ are mergeable (with
% $v_i$'s locus as the LCA), and so do $v$ and $v_j$. If $v_i$ and $v_j$
% are the latest versions on their respective branches, then merging $v$
% into $v_i$ and $v$ into $v_j$ makes $v$ the locus of both the
% branches, thus making them mergeable again.

% \begin{theorem} [\bfseries Progress and Eventual Convergence] Given a
% legal branching history $H$, and a program $p$, where every thread
% expression in $p$ is a \C{pull}, there exist a series of reduction steps
% that reduce $p; H$ to $p'; H'$, where every thread expression in $p$
% is the same value $v$.
% \end{theorem}

