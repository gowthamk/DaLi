\section{Related Work}
\label{sec:related}

Our idea of versioning state bears resemblance to Concurrent
Revisions~\cite{BBL+10}, a programming abstraction that provides
deterministic concurrent execution by forking and merging
\emph{versions}, portions of state shared among concurrent threads.
The idea of using revisions as a means to programming eventually
consistent distributed systems was further developed
in~\cite{BFL+12,Burckhardt2012}.  The \name\ programming model,
however, differs substantially from a concurrent revisions model
because it imposes no distinction between \emph{servers}, machines
that hold global state, and \emph{clients}, devices that operate on
local, potentially stale, data - any computation executing in a
distributed environment is free to fork new versions, and synchronize
against other replicated state.  Our model, furthermore, supports
fully decentralized operation and is robust to network partitions and
failures. Just as significantly, \name\ allows applications to
customize join semantics with programmable merge operations.  Indeed,
the integration of a version-based mechanism within OCaml allows a
degree of type safety, composability, and profitable use of
polymorphism not available in related systems.

\cite{Burckhardt2015} also presents an operational model of a
replicated data store that is based on the abstract system model
presented in ~\cite{Burckhardt2014}; their design is similar to the
model described in~\cite{pldi15}.  In both approaches, coordination
among replicas involves transmitting operations on replicated objects
that are performed locally on each replica.  In contrast,
\name\ allows programmers to use familiar state-based and functional
abstractions when developing distributed applications.  As we
illustrated in Fig.~\ref{fig:counter-rdt}, using effects and
operations to coordinate the activities of replicas may involve
addressing subtleties that do not manifest otherwise.  Our case
studies and experimental results support our contention that using
well-understood state (heap)-based abstractions to build distributed
applications greatly simplifies program reasoning and eases
development, without compromising efficiency.

Modern distributed systems are often equipped with only parsimonious
data models (e.g., key-value model) and poorly understood low-level
consistency guarantees\footnote{Cassandra~\cite{Cassandra}, a popular
  NoSQL data store, comes with various consistency enforcement
  mechanisms, such as anti-entropy protocols, {\sc QUORUM} and {\sc
    LOCAL\_QUORUM} reads and writes, and light-weight transactions,
  each of which can be controlled via configuration knobs or runtime
  parameters.}  that complicate program reasoning, and make it hard to
enforce application integrity. Some authors~\cite{bailis-bolton} have
demonstrated that it is possible to\emph{bolt on} high-level
consistency guarantees (e.g., causal consistency)~\cite{COPS,BEG+17}
as a \emph{shim layer} service over existing stores.

A number of verification techniques, programming abstractions, and
tools have been proposed to reason about program behavior in a
geo-replicated weakly consistent environment.  These techniques treat
replicated storage as a black box with a fixed pre-defined consistency
model~\cite{bailis-vldb, alvaro-calm, gotsman-popl16,redblue-atc,
  redblue-osdi, ecinec}.  On the other hand, compositional proof
techniques and mechanized verification frameworks have been developed
to rigorously reason about various components of a distributed data
store~\cite{verdi,lbc16}. \name seeks to provide a rich high-level
programming model, built on rigorous foundations, that can facilitate
program reasoning and verification.  An important by-product of the
programming model is that it does not require algorithmic
restructuring to transplant a sequential or concurrent program to a
distributed, replicated setting; the only additional burden imposed on
the developer is the need to provide a merge operator, a function that
can be often easily written for many common datatypes.

Several conditions have been proposed to judge whether an operation on
a replicated data object needs coordination or not. ~\cite{Calm}
defines \emph{logical monotonicity} as a sufficient condition for
coordination freedom, and proposes a consistency analysis that marks
code regions performing non-monotonic reasoning (eg: aggregations,
such as \C{COUNT}) as potential coordination
points. ~\cite{IConfluence} and ~\cite{Sieve} define \emph{invariant
  confluence} and \emph{invariant safety}, respectively, as conditions
for safely executing an operation without coordination.
~\cite{indigo} requires programmers to declare application semantics,
and the desired application-specific invariants as formulas in
first-order logic. It performs static analysis on these formulas to
determine $I$-offender sets - sets of operations, which, when
performed concurrently, result in violation of one or more of the
stated invariants. For each offending set of operations, if the
programmer chooses invariant-violation avoidance over violation
repair, the system employs various techniques, such as escrow
reservation, to ensure that the offending set is effectively
serialized.  All of these approaches differ significantly from the
core goals of \name, which is to enable seamless application-driven
techniques for programming geo-replicated systems.  The
\name\ programmer is not concerned with different system-specific
consistency or isolation levels, or invariants that are sensitive to
these levels.  Instead, the only requirements demanded of the
developer is the need to reason about a sensible merge semantics for
data structures.  Such reasoning can be applied without consideration
of system- or architecture-specific details.  This reasoning phase
implicitly expresses salient semantic invariants without having to be
exposed to low-level operational details.

There have been numerous proposals over the years for realizing
distributed functionality in a functional programming context.
Poly/ML~\cite{Mat97} and Facile~\cite{TLK+93} both extend Standard ML
with distributed programming abstractions.  Kali-Scheme~\cite{CJK95}
defines a distributed extension for Scheme-48, and Cloud
Haskell~\cite{EBPJ11} describes a domain-specific language that
enables development of distributed Haskell programs.
Obliq~\cite{Car95} explores issues related to mobility, lexical
scoping, and network references in an object-oriented framework that
enables safe (i.e., lexical) transmission of closures.  However,
unlike \name, none of these systems consider issues of object
replication or consistency as central to their programming models.
The realities of modern-day scalable distributed systems, especially
those that are constructed from geographically distributed
participants, dictate that we must necessarily grapple with
consistency issues to achieve any kind of reasonable performance; it
is noteworthy that replication is inherent in many widely-used
distributed system implementations today (e.g.,
Cassandra~\cite{Cassandra} or Dynamo~\cite{Dynamo}), which provide
only weak consistency guarantees.  Rather than providing, as other
approaches have, a low-level operational treatment of replication, our
primary contribution is the development of a principled high-level
approach to framing issues of consistency and replication.  Our goal
is to leverage functional programming principles to minimize
disruption to the way programmers structure and reason about their
applications, without compromising efficiency or performance.  Indeed,
the semantics of mergeable types and versioning has utility in any
concurrent setting that must deal with non-trivial coordination and
synchronization costs, even without taking distribution or replication
issues into account.
